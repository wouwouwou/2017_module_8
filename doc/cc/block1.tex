% !TEX TS-program = pdflatex
% !TEX encoding = UTF-8 Unicode

% This is a simple template for a LaTeX document using the "article" class.
% See "book", "report", "letter" for other types of document.

\documentclass[11pt]{article} % use larger type; default would be 10pt

\usepackage[utf8]{inputenc} % set input encoding (not needed with XeLaTeX)

%%% Examples of Article customizations
% These packages are optional, depending whether you want the features they provide.
% See the LaTeX Companion or other references for full information.

%%% PAGE DIMENSIONS
\usepackage{geometry} % to change the page dimensions
\geometry{a4paper} % or letterpaper (US) or a5paper or....
% \geometry{margin=2in} % for example, change the margins to 2 inches all round
% \geometry{landscape} % set up the page for landscape
%   read geometry.pdf for detailed page layout information

\usepackage{graphicx} % support the \includegraphics command and options

% \usepackage[parfill]{parskip} % Activate to begin paragraphs with an empty line rather than an indent

%%% PACKAGES
\usepackage{booktabs} % for much better looking tables
\usepackage{array} % for better arrays (eg matrices) in maths
\usepackage{paralist} % very flexible & customisable lists (eg. enumerate/itemize, etc.)
\usepackage{verbatim} % adds environment for commenting out blocks of text & for better verbatim
\usepackage{subfig} % make it possible to include more than one captioned figure/table in a single float
% These packages are all incorporated in the memoir class to one degree or another...

%%% HEADERS & FOOTERS
\usepackage{fancyhdr} % This should be set AFTER setting up the page geometry
\pagestyle{fancy} % options: empty , plain , fancy
\renewcommand{\headrulewidth}{0pt} % customise the layout...
\lhead{}\chead{}\rhead{}
\lfoot{}\cfoot{\thepage}\rfoot{}

%%% SECTION TITLE APPEARANCE
\usepackage{sectsty}
\allsectionsfont{\sffamily\mdseries\upshape} % (See the fntguide.pdf for font help)
% (This matches ConTeXt defaults)

%%% ToC (table of contents) APPEARANCE
\usepackage[nottoc,notlof,notlot]{tocbibind} % Put the bibliography in the ToC
\usepackage[titles,subfigure]{tocloft} % Alter the style of the Table of Contents
\renewcommand{\cftsecfont}{\rmfamily\mdseries\upshape}
\renewcommand{\cftsecpagefont}{\rmfamily\mdseries\upshape} % No bold!

%%% END Article customizations

%%% The "real" document content comes below...

\title{Exercises Compiler Construction}
\author{Martijn Verkleij (s...) \& Wouter Bos (s1606824)}
%\date{} % Activate to display a given date or no date (if empty),
         % otherwise the current date is printed 

\begin{document}
\maketitle

\section*{Exercise 1}
\begin{tabular}{ll}
back-end					& maps code into computer specific code (thrid phase of a compiler)				\\
front-end					& understands code syntax and checks for errors (firts phase of a compiler)		\\
grammar						& rules of a language, used in front-end after scanning (parsing part)			\\
instruction scheduling		& choosing the order of the instructions										\\
instruction selection		& choosing which instructions to use											\\
optimiser					& a transformer that improves the IR											\\
parsing						& grouping of tokens based on a grammar, second phase of front-end				\\
register allocation			& optimises the registers used in the program									\\
scanning					& transforms the (input) code language into tokens, first phase of front-end	\\
type checking				& checking if the groups of tokens are meaningful, third phase of front-end		\\
\end{tabular}

\section*{Exercise 2.1}
\begin{tabular}{|c|c|c|c|c|c|}	\hline
Most 				& students 			& is 	& good 					& programmers 	& . 		\\\hline
adj 				& noun 				& verb 	& adj 					& noun 			& end 		\\\hline
\textbf{Modifier}	& noun				& verb	& \textbf{Modifier}		& noun			& end		\\\hline
\multicolumn{2}{|c|}{\textbf{Subject}}	& verb 	& \multicolumn{2}{c|}{\textbf{Object}}	& end		\\\hline
\multicolumn{6}{|c|}{\textbf{Sentence}}																\\\hline
\end{tabular}

\section*{Exercise 2.2}
\textit{is} should be \textit{are} because \textit{students} is plural. This could be compared to / is the equivalent
of a type error.

\section*{Exercise 2.3}
Parsing and type checking

\section*{Exercise 3}

\section*{Exercise 4.1}

\section*{Exercise 4.2}

\section*{Exercise 4.3}

\section*{Exercise 5}

\section*{Exercise 6}

\section*{Exercise 7.1}

\section*{Exercise 7.2}

\section*{Exercise 7.3}

More text.

\end{document}
