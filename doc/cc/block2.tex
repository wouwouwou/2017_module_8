% !TEX TS-program = pdflatex
% !TEX encoding = UTF-8 Unicode

% This is a simple template for a LaTeX document using the "article" class.
% See "book", "report", "letter" for other types of document.

\documentclass[11pt]{article} % use larger type; default would be 10pt

\usepackage[utf8]{inputenc} % set input encoding (not needed with XeLaTeX)

%%% Examples of Article customizations
% These packages are optional, depending whether you want the features they provide.
% See the LaTeX Companion or other references for full information.

%%% PAGE DIMENSIONS
\usepackage{geometry} % to change the page dimensions
\geometry{a4paper} % or letterpaper (US) or a5paper or....
% \geometry{margin=2in} % for example, change the margins to 2 inches all round
% \geometry{landscape} % set up the page for landscape
%   read geometry.pdf for detailed page layout information

\usepackage{graphicx} % support the \includegraphics command and options
\usepackage{float}

% \usepackage[parfill]{parskip} % Activate to begin paragraphs with an empty line rather than an indent

%%% PACKAGES
\usepackage{booktabs} % for much better looking tables
\usepackage{array} % for better arrays (eg matrices) in maths
\usepackage{paralist} % very flexible & customisable lists (eg. enumerate/itemize, etc.)
\usepackage{verbatim} % adds environment for commenting out blocks of text & for better verbatim
\usepackage{subfig} % make it possible to include more than one captioned figure/table in a single float
% These packages are all incorporated in the memoir class to one degree or another...

%%% HEADERS & FOOTERS
\usepackage{fancyhdr} % This should be set AFTER setting up the page geometry
\pagestyle{fancy} % options: empty , plain , fancy
\renewcommand{\headrulewidth}{0pt} % customise the layout...
\lhead{}\chead{}\rhead{}
\lfoot{}\cfoot{\thepage}\rfoot{}

%%% SECTION TITLE APPEARANCE
\usepackage{sectsty}
\allsectionsfont{\sffamily\mdseries\upshape} % (See the fntguide.pdf for font help)
% (This matches ConTeXt defaults)

%%% ToC (table of contents) APPEARANCE
\usepackage[nottoc,notlof,notlot]{tocbibind} % Put the bibliography in the ToC
\usepackage[titles,subfigure]{tocloft} % Alter the style of the Table of Contents
\renewcommand{\cftsecfont}{\rmfamily\mdseries\upshape}
\renewcommand{\cftsecpagefont}{\rmfamily\mdseries\upshape} % No bold!

%% Automata
\usepackage{pgf}
\usepackage{tikz}
\usetikzlibrary{arrows,automata,positioning}
\usepackage{tikz-qtree}
%%% END Article customizations

%%% The "real" document content comes below...

\title{Exercises Compiler Construction}
\author{Martijn Verkleij (s1466895) \& Wouter Bos (s1606824)}
%\date{} % Activate to display a given date or no date (if empty),
         % otherwise the current date is printed 

\begin{document}
\maketitle

\section*{Exercise 1}
Calculation of the FIRST set of all Terminals:\\

\begin{tabular}{|c|c|c|c|c|c|c|} \hline 
\textbf{assign}
& \textbf{if}
& \textbf{else}
& \textbf{then}
& \textbf{expr}
& \textbf{$\epsilon$}
& \textbf{EOF} \\\hline 

assign
& if
& else
& then
& expr
& $\epsilon$
& EOF \\\hline 
\end{tabular} \\

\noindent Calculation of the FIRST set of all Non-Terminals:\\

\begin{tabular}{|p{2cm}|c|} \hline 
\textbf{Stat}
& \textbf{ElsePart} \\\hline 

& \\\hline 
\end{tabular} \\

\begin{tabular}{|p{2cm}|c|} \hline 
\textbf{Stat}
& \textbf{ElsePart} \\\hline 

assign
& \\\hline 
\end{tabular} \\

\begin{tabular}{|p{2cm}|c|} \hline 
\textbf{Stat}
& \textbf{ElsePart} \\\hline 

assign, if
& else \\\hline 
\end{tabular} \\

\begin{tabular}{|p{2cm}|c|} \hline 
\textbf{Stat}
& \textbf{ElsePart} \\\hline

assign, if
& else, $\epsilon$ \\\hline 
\end{tabular}\\

\noindent Calculation of the FOLLOW set:\\

\begin{tabular}{|p{2cm}|c|} \hline 
\textbf{Stat}
& \textbf{ElsePart} \\\hline 

& \\\hline 
\end{tabular} \\

\begin{tabular}{|p{2cm}|c|} \hline 
\textbf{Stat}
& \textbf{ElsePart} \\\hline

EOF
& \\\hline 
\end{tabular} \\

\begin{tabular}{|p{2cm}|c|} \hline 
\textbf{Stat}
& \textbf{ElsePart} \\\hline

EOF
& EOF \\\hline 
\end{tabular} \\

\begin{tabular}{|p{2cm}|c|} \hline 
\textbf{Stat}
& \textbf{ElsePart} \\\hline 

EOF, else
& EOF, else \\\hline 
\end{tabular}\\

\noindent Calculation of the FIRST+ set (only rule 4 is necessary):\\

\begin{tabular}{|c|c|c|c|} \hline 
& Production
& FIRST Set
& FIRST+ Set \\\hline

4
& \textbf{ElsePart} $\rightarrow \epsilon$
& \{$\epsilon$\}
& \{$\epsilon$, EOF, else\} \\\hline 
\end{tabular}\\

\noindent The FIRST+ set of \textbf{ElsePart} $\rightarrow \epsilon$ (\{$\epsilon$, EOF, else\}) is not disjoint with FIRST+ set of \textbf{ElsePart} $\rightarrow$ else Stat (\{else, $\epsilon$\}). Therefore, the grammar does not meet the criteria for \textit{LL(1)}

\section*{Excercise 2}
\begin{tabular}{llcl}
1
& \textbf{L}
& $\rightarrow$
& \textbf{R}a \\

2
& 
& $\vert$
& \textbf{Q}ba \\

3
& \textbf{R}
& $\rightarrow$
& aba \\

4
& 
& $\vert$
& caba \\

5
& 
& $\vert$
& \textbf{R}bc \\

6
& \textbf{Q}
& $\rightarrow$
& bbc \\

7
& 
& $\vert$
& bc \\
\end{tabular}\\

\noindent FIRST of Terminals: \\

\begin{tabular}{|c|c|c|c|c|} \hline 
\textbf{a}
& \textbf{b}
& \textbf{c}
& \textbf{EOF}
& \textbf{$\epsilon$}\\\hline 

a
& b
& c
& EOF
& $\epsilon$\\\hline  
\end{tabular} \\

\noindent FIRST of Non-Terminals: \\

\begin{tabular}{|c|c|c|} \hline 
\textbf{L}
& \textbf{R}
& \textbf{Q} \\\hline 

a, b, c
& a, c
& b \\\hline  
\end{tabular} \\

\noindent FOLLOW: \\

\begin{tabular}{|c|c|c|} \hline 
\textbf{L}
& \textbf{R}
& \textbf{Q} \\\hline 

EOF
& EOF, a, b, c
& EOF, a, b \\\hline  
\end{tabular} \\

\noindent Problem: left recursion at rule 5 and double b at rules 6 and 7!  \\

\begin{tabular}{llcl}
1
& \textbf{L}
& $\rightarrow$
& \textbf{R}a \\

2
& 
& $\vert$
& \textbf{Q}ba \\

3
& \textbf{R}
& $\rightarrow$
& aba\textbf{R'} \\

4
& 
& $\vert$
& caba\textbf{R'} \\

6
& \textbf{Q}
& $\rightarrow$
& b\textbf{Q'} \\

8
& \textbf{R'}
& $\rightarrow$
& bc\textbf{R'} \\

9
& 
& $\vert$
& $\epsilon$ \\

10
& \textbf{Q'}
& $\rightarrow$
& bc \\

11
& 
& $\vert$
& c \\
\end{tabular} \\

\noindent Showing that the new grammar satisfies the \textit{LL(1)} condition: \\

\noindent FIRST of Terminals: \\

\begin{tabular}{|c|c|c|c|c|} \hline 
\textbf{a}
& \textbf{b}
& \textbf{c}
& \textbf{EOF}
& \textbf{$\epsilon$}\\\hline 

a
& b
& c
& EOF
& $\epsilon$\\\hline  
\end{tabular} \\

\noindent FIRST of Non-Terminals: \\

\begin{tabular}{|c|c|c|c|c|} \hline 
\textbf{L}
& \textbf{R}
& \textbf{Q}
& \textbf{R'}
& \textbf{Q'} \\\hline 

a, b, c
& a, c
& b
& b, $\epsilon$
& b, c \\\hline  
\end{tabular} \\

\noindent FOLLOW: \\

\begin{tabular}{|c|c|c|c|c|} \hline 
\textbf{L}
& \textbf{R}
& \textbf{Q}
& \textbf{R'}
& \textbf{Q'} \\\hline

EOF
& a
& b
& a
& b \\\hline  
\end{tabular} \\

\noindent $\epsilon$ in rule 9, FIRST+ set: \\

\begin{tabular}{|c|c|c|c|} \hline 
& Production
& FIRST Set
& FIRST+ Set \\\hline

9
& \textbf{R'} $\rightarrow \epsilon$
& \{$\epsilon$\}
& \{$\epsilon$, a\} \\\hline 
\end{tabular}\\

\noindent Every FIRST+ set of every Non-Terminal rule-combination is disjoint, so this satisfies the \textit{LL(1)} condition!


\section*{Exercise 7}
The parsed sentence is printed to stdout. The "tree" is visible from the printed structure, see also the tree exercise in the answers of block 1. Number 3 of the rightmost derivation looks equal to the parsed sentence. If the sentence is invalid, the program prints a friendly error and also adds the way the sentence has been parsed.

\section*{Exercise 9}
1) Not only the entry/exit methods are specified in two alternatives, also the parameters are specified in those alternatives.
Instead of a SubjectContext we now have a ModSubjectContext and a SimpleSubjectContext which is imported from the parser.
These classes extend the SubjectContext class and therefore inherit everything from it. However, their fields and
methods are following the rule alternative they are for. (fields modifier subject vs noun, methods which use those fields).\\

\noindent 2) It corresponds more with the rightmost parse tree now, because the assoc now results in a rightmost derivation of the modifiers.
If you leave the assoc it will go back to a leftmost parse tree.

\end{document}
