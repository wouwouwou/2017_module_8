%------------------------------------------------
%	PACKAGES AND OTHER DOCUMENT CONFIGURATIONS
%------------------------------------------------

\documentclass[twoside]{report}

\usepackage{graphicx}

\usepackage{minted}

\usepackage{amsmath,amssymb,amsthm} % Mathematical Symbols, styles, etc

\usepackage[sc]{mathpazo} % Use the Palatino font
% Output encoding
\usepackage[T1]{fontenc} % Use 8-bit encoding that has 256 glyphs
% Input encoding
\usepackage[utf8]{inputenc} % UTF-8 character encoding stuff
\linespread{1.05} % Line spacing - Palatino needs more space between lines
\usepackage{microtype} % Slightly tweak font spacing for aesthetics

\usepackage[hmarginratio=1:1,top=32mm,columnsep=20pt]{geometry} % Document margins
\usepackage[hang, small,labelfont=bf,up,textfont=it,up]{caption} % Custom captions under/above floats in tables or figures
\usepackage{booktabs} % Horizontal rules in tables
\usepackage{float} % Required for tables and figures in the multi-column environment - they need to be placed in specific locations with the [H] (e.g. \begin{table}[H])
\usepackage{hyperref} % For hyperlinks in the PDF
\newcommand*{\fullref}[1]{\hyperref[{#1}]{\ref*{#1} (\nameref*{#1})}}
\newcommand*{\fullautoref}[1]{\hyperref[{#1}]{\autoref*{#1} (\nameref*{#1})}}

\usepackage{pdflscape} % For landscape pages

\usepackage{lettrine} % The lettrine is the first enlarged letter at the beginning of the text
\usepackage{paralist} % Used for the compactitem environment which makes bullet points with less space between them

\usepackage{titlesec} % Allows customization of titles
%\renewcommand\thesection{\Roman{section}} % Roman numerals for the sections
%\renewcommand\thesubsection{\Roman{subsection}} % Roman numerals for subsections
%\titleformat{\section}[block]{\large\scshape}{\thesection.}{1em}{} % Change the look of the section titles
%\titleformat{\subsection}[block]{\large}{\thesubsection.}{1em}{} % Change the look of the subsection titles

\usepackage{fancyhdr} % Headers and footers
\pagestyle{fancy} % All pages have headers and footers
\fancyhead{} % Blank out the default header
\fancyfoot{} % Blank out the default footer
\fancyhead[C]{M.F. Verkleij, T. Kerkhoven: \shorttitle} % Custom header text
\fancyfoot[RO,LE]{\thepage} % Custom footer text

% Appendices
\usepackage[toc,page]{appendix} % appendix

% Additional column type
\usepackage{array}
\newcolumntype{C}[1]{>{\centering\arraybackslash}p{#1}}

% Indentation of list
\usepackage{changepage}
\newenvironment{mycompactdesc}{\begin{adjustwidth}{0.53cm}{}\begin{compactdesc}}{\end{compactdesc}\end{adjustwidth}}

%------------------------------------------------
%	TITLE SECTION
%------------------------------------------------

\newcommand{\articletitle}{Programming Paradigms Final Project: Test Report}
\newcommand{\shorttitle}{Test Report}

\title{\vspace{-15mm}\fontsize{24pt}{10pt}\selectfont\textbf{\articletitle}} % Article title

\author{
\large
\textsc{Group 26}\\[-0.75mm]
\textsc{Martijn Verkleij \& Tim Kerkhoven}\\[2mm] % Your name
\normalsize University of Twente \\ % Your institution
\normalsize \href{mailto:m.f.verkleij@student.utwente.nl}{m.f.verkleij@student.utwente.nl},
\href{mailto:t.kerkhoven@student.utwente.nl}{t.kerkhoven@student.utwente.nl}\\% Your email addresses
\normalsize s1466895 s1375253
}

\date{\today}

%------------------------------------------------

\begin{document}

\thispagestyle{empty}
\maketitle % Insert title


%------------------------------------------------
%	ARTICLE CONTENTS
%------------------------------------------------

%------------------------------------------------
\tableofcontents


%------------------------------------------------
\chapter{Syntax Tests}

\section{Syntax 1}
\subsection{Source}
\inputminted[tabsize=4,linenos,firstnumber=1]{text}{../../src/haskell/PP-project-2016/test/syntax1.shl}
\subsection{Output}
\begin{minted}[tabsize=4,linenos]{text}
Main: tokenList not fully parsed
\end{minted}

\section{Syntax 2}
\subsection{Source}
\inputminted[tabsize=4,linenos,firstnumber=1]{text}{../../src/haskell/PP-project-2016/test/syntax2.shl}
\subsection{Output}
\begin{minted}[tabsize=4,linenos]{text}
Main: tokenList not fully parsed
\end{minted}

\section{Syntax 3}
\subsection{Source}
\inputminted[tabsize=4,linenos,firstnumber=1]{text}{../../src/haskell/PP-project-2016/test/syntax3.shl}
\subsection{Output}
\begin{minted}[tabsize=4,linenos]{text}
Main: tokenList not fully parsed
\end{minted}

\section{Syntax 4}
\subsection{Source}
\inputminted[tabsize=4,linenos,firstnumber=1]{text}{../../src/haskell/PP-project-2016/test/syntax4.shl}
\subsection{Output}
\begin{minted}[tabsize=4,linenos]{text}
Main: tokenList not fully parsed
\end{minted}

\section{Syntax 5}
\subsection{Source}
\inputminted[tabsize=4,linenos,firstnumber=1]{text}{../../src/haskell/PP-project-2016/test/syntax5.shl}
\subsection{Output}
\begin{minted}[tabsize=4,linenos]{text}
Main: tokenList not fully parsed
\end{minted}


\chapter{Contextual Tests}

\section{Wrong Type}
\subsection{Source}
\inputminted[tabsize=4,linenos,firstnumber=1]{text}{../../src/haskell/PP-project-2016/test/wrong_type.shl}
\subsection{Output}
\begin{minted}[tabsize=4,linenos]{text}
Main: Condition in if statement should be of type: bool, but isnt, in: ASTVar "i" ([],[],[])
\end{minted}

\section{Not Declared}
\subsection{Source}
\inputminted[tabsize=4,linenos,firstnumber=1]{text}{../../src/haskell/PP-project-2016/test/not_declared.shl}
\subsection{Output}
\begin{minted}[tabsize=4,linenos]{text}
Main: Variable: i not declared in Checker.getExprType.iterVar
\end{minted}


\chapter{Semantic Tests}

\section{Banking}
\subsection{Source}
\inputminted[tabsize=4,linenos,firstnumber=1]{text}{../../src/haskell/PP-project-2016/test/banking.shl}
\subsection{Generated SprIL}
\inputminted[tabsize=4,linenos,firstnumber=0]{text}{../../src/haskell/PP-project-2016/test/banking_gen.txt}
\subsection{Results}
\begin{minted}[tabsize=4,linenos]{text}
>>> 10000
>>> 2000
>>> 99999
>>> 10100
>>> 2100
>>> 100000
>>> 9900
>>> 2300
>>> 102000
>>> 9880
>>> 2290
>>> 102100
>>> 9880
>>> 1990
>>> 102050
>>> 10880
>>> 1955
>>> 101050
>>> 10880
>>> 2055
>>> 100950
\end{minted}

\section{Blocks}
\subsection{Source}
\inputminted[tabsize=4,linenos,firstnumber=1]{text}{../../src/haskell/PP-project-2016/test/blocks.shl}
\subsection{Generated SprIL}
\inputminted[tabsize=4,linenos,firstnumber=0]{text}{../../src/haskell/PP-project-2016/test/blocks_gen.txt}
\subsection{Results}
\begin{minted}[tabsize=4,linenos]{text}
>>> 1
>>> 100
>>> 0
>>> 0
>>> 2
>>> 120
>>> 1
>>> 0
>>> 3
>>> 123
>>> 0
>>> 1
>>> 4
>>> 423
>>> 1
>>> 1
>>> 3
>>> 123
>>> 0
>>> 1
>>> 5
>>> 453
>>> 1
>>> 0
>>> 5
>>> 453
>>> 0
>>> 1
>>> 5
>>> 453
>>> 1
>>> 0
>>> 6
>>> 456
>>> 0
>>> 0
>>> 5
>>> 453
>>> 1
>>> 0
>>> 3
>>> 123
>>> 0
>>> 1
>>> 2
>>> 120
>>> 1
>>> 0
>>> 2
>>> 120
>>> 1
>>> 0
>>> 2
>>> 120
>>> 1
>>> 0
>>> 2
>>> 120
>>> 1
>>> 0
>>> 1
>>> 100
>>> 0
>>> 0
\end{minted}

\section{Call-by-reference}
\subsection{Source}
\inputminted[tabsize=4,linenos,firstnumber=1]{text}{../../src/haskell/PP-project-2016/test/call_by_reference.shl}
\subsection{Generated SprIL}
\inputminted[tabsize=4,linenos,firstnumber=0]{text}{../../src/haskell/PP-project-2016/test/call_by_reference_gen.txt}
\subsection{Results}
\begin{minted}[tabsize=4,linenos]{text}
>>> 42
>>> 1337
\end{minted}

\section{Cyclic Recursion}
\subsection{Source}
\inputminted[tabsize=4,linenos,firstnumber=1]{text}{../../src/haskell/PP-project-2016/test/cyclic_recursion.shl}
\subsection{Generated SprIL}
\inputminted[tabsize=4,linenos,firstnumber=0]{text}{../../src/haskell/PP-project-2016/test/cyclic_recursion_gen.txt}
\subsection{Results}
\begin{minted}[tabsize=4,linenos]{text}
>>> 17
>>> 16
>>> 15
>>> 14
>>> 13
>>> 12
>>> 11
>>> 10
>>> 9
>>> 8
>>> 7
>>> 6
>>> 5
>>> 4
>>> 3
>>> 2
>>> 1
>>> 0
>>> 0
\end{minted}

\section{Deep Expression}
\subsection{Source}
\inputminted[tabsize=4,linenos,firstnumber=1]{text}{../../src/haskell/PP-project-2016/test/deep_expression.shl}
\subsection{Generated SprIL}
\inputminted[tabsize=4,linenos,firstnumber=0]{text}{../../src/haskell/PP-project-2016/test/deep_expression_gen.txt}
\subsection{Results}
\begin{minted}[tabsize=4,linenos]{text}
>>> 5024000
\end{minted}

\section{Fib}
\subsection{Source}
\inputminted[tabsize=4,linenos,firstnumber=1]{text}{../../src/haskell/PP-project-2016/test/fib.shl}
\subsection{Generated SprIL}
\inputminted[tabsize=4,linenos,firstnumber=0]{text}{../../src/haskell/PP-project-2016/test/fib_gen.txt}
\subsection{Results}
\begin{minted}[tabsize=4,linenos]{text}
>>> 21
\end{minted}

\section{If}
\subsection{Source}
\inputminted[tabsize=4,linenos,firstnumber=1]{text}{../../src/haskell/PP-project-2016/test/if.shl}
\subsection{Generated SprIL}
\inputminted[tabsize=4,linenos,firstnumber=0]{text}{../../src/haskell/PP-project-2016/test/if_gen.txt}
\subsection{Results}
\begin{minted}[tabsize=4,linenos]{text}
>>> 1
\end{minted}

\section{If Else}
\subsection{Source}
\inputminted[tabsize=4,linenos,firstnumber=1]{text}{../../src/haskell/PP-project-2016/test/ifelse.shl}
\subsection{Generated SprIL}
\inputminted[tabsize=4,linenos,firstnumber=0]{text}{../../src/haskell/PP-project-2016/test/ifelse_gen.txt}
\subsection{Results}
\begin{minted}[tabsize=4,linenos]{text}
>>> 4
>>> 5
>>> 4
>>> 3
>>> 4
\end{minted}

\section{Infinite Busy Loop}
\subsection{Source}
\inputminted[tabsize=4,linenos,firstnumber=1]{text}{../../src/haskell/PP-project-2016/test/infinite_busy_loop.shl}
\subsection{Generated SprIL}
\inputminted[tabsize=4,linenos,firstnumber=0]{text}{../../src/haskell/PP-project-2016/test/infinite_busy_loop_gen.txt}
\subsection{Results}
Gets stuck in an infinite loop, repeating the same output.
\begin{minted}[tabsize=4,linenos]{text}
>>> 1
>>> 1
>>> 2
>>> 2
>>> 4
>>> 8
>>> 12
>>> 96
>>> 108
>>> 10368
>>> 10476
>>> 108615168
>>> 108625644
>>> 11798392572168192
>>> 11798392680793836
>>> -5570361874949185536
>>> -5558563482268391700
>>> 3671369242980155392
>>> -1887194239288236308
>>> -4483044364780175360
>>> -6370238604068411668
>>> -8730959061097906176
>>> 3345546408543233772
>>> -6745737849034768384
>>> -3400191440491534612
>>> -6096120617457680384
>>> 8950432015760336620
>>> -1019520187243692032
>>> 7930911828516644588
>>> -4809903748681826304
>>> 3121008079834818284
>>> 5865085819223539712
>>> 8986093899058357996
>>> 2740241432517279744
>>> -6720408742133913876
>>> 3246081813541552128
>>> -3474326928592361748
>>> -1859074291971129344
>>> -5333401220563491092
>>> 681350175863603200
>>> -4652051044699887892
>>> -8143132099134619648
>>> 5651560929875044076
>>> 6951259845357993984
>>> -5843923298476513556
>>> 4700992750881865728
>>> -1142930547594647828
>>> -8561800288468467712
>>> 8742013237646436076
>>> 7566188111470788608
>>> -2138542724592326932
>>> -6956372574427152384
>>> -9094915299019479316
>>> -8878846665360932864
>>> 472982109329139436
>>> 1756403854674493440
>>> 2229385964003632876
>>> -5152117973711847424
>>> -2922732009708214548
>>> 1585267068834414592
>>> -1337464940873799956
>>> 5188146770730811392
>>> 3850681829857011436
>>> 6917529027641081856
>>> -7678533216211458324
>>> -9223372036854775808
>>> 1544838820643317484
>>> 0
>>> 1544838820643317484
>>> 0
>>> 1544838820643317484
>>> 0
>>> 1544838820643317484
>>> 0
\ldots
\end{minted}


\section{Infinite Empty Loop}
\subsection{Source}
\inputminted[tabsize=4,linenos,firstnumber=1]{text}{../../src/haskell/PP-project-2016/test/infinite_loop.shl}
\subsection{Generated SprIL}
\inputminted[tabsize=4,linenos,firstnumber=0]{text}{../../src/haskell/PP-project-2016/test/infinite_loop_gen.txt}
\subsection{Results}
No output, gets stuck in an infinite loop.

\section{Join Test}
\subsection{Source}
\inputminted[tabsize=4,linenos,firstnumber=1]{text}{../../src/haskell/PP-project-2016/test/join_test.shl}
\subsection{Generated SprIL}
\inputminted[tabsize=4,linenos,firstnumber=0]{text}{../../src/haskell/PP-project-2016/test/join_test_gen.txt}
\subsection{Results}
No output, gets stuck in an infinite loop.

\section{Multiple Globals}
\subsection{Source}
\inputminted[tabsize=4,linenos,firstnumber=1]{text}{../../src/haskell/PP-project-2016/test/multiple_globals.shl}
\subsection{Generated SprIL}
\inputminted[tabsize=4,linenos,firstnumber=0]{text}{../../src/haskell/PP-project-2016/test/multiple_globals_gen.txt}
\subsection{Results}
\begin{minted}[tabsize=4,linenos]{text}
>>> 8
>>> 9
>>> 10
>>> 11
>>> 12
>>> 13
>>> 13
>>> 12
>>> 11
>>> 10
>>> 9
>>> 8
\end{minted}

\section{Nested Procedures}
\subsection{Source}
\inputminted[tabsize=4,linenos,firstnumber=1]{text}{../../src/haskell/PP-project-2016/test/nested_procedures.shl}
\subsection{Generated SprIL}
\inputminted[tabsize=4,linenos,firstnumber=0]{text}{../../src/haskell/PP-project-2016/test/nested_procedures_gen.txt}
\subsection{Results}
\begin{minted}[tabsize=4,linenos]{text}
>>> 90
>>> 10
>>> 20
>>> 30
>>> 40
>>> 34
>>> 23
>>> 40
>>> 24
>>> 12
>>> 30
>>> 40
>>> 34
>>> 13
>>> 40
>>> 14
>>> 91
>>> 20
>>> 30
>>> 40
>>> 34
>>> 23
>>> 40
>>> 24
>>> 92
>>> 30
>>> 40
>>> 34
>>> 93
>>> 40
>>> 94
\end{minted}

\section{Peterson}
\subsection{Source}
\inputminted[tabsize=4,linenos,firstnumber=1]{text}{../../src/haskell/PP-project-2016/test/peterson.shl}
\subsection{Generated SprIL}
\inputminted[tabsize=4,linenos,firstnumber=0]{text}{../../src/haskell/PP-project-2016/test/peterson_gen.txt}
\subsection{Results}
\begin{minted}[tabsize=4,linenos]{text}
>>> 0
>>> 0
>>> 0
>>> 0
>>> 0
>>> 0
>>> 0
>>> 0
>>> 0
>>> 0
>>> 0
>>> 0
>>> 0
>>> 0
>>> 0
>>> 0
>>> 0
>>> 0
>>> 0
>>> 0
\end{minted}

\section{Recursion}
\subsection{Source}
\inputminted[tabsize=4,linenos,firstnumber=1]{text}{../../src/haskell/PP-project-2016/test/recursion.shl}
\subsection{Generated SprIL}
\inputminted[tabsize=4,linenos,firstnumber=0]{text}{../../src/haskell/PP-project-2016/test/recursion_gen.txt}
\subsection{Results}
\begin{minted}[tabsize=4,linenos]{text}
>>> 0
>>> 1
>>> 2
>>> 3
\end{minted}

\section{Simple Concurrency}
\subsection{Source}
\inputminted[tabsize=4,linenos,firstnumber=1]{text}{../../src/haskell/PP-project-2016/test/simple_concurrency.shl}
\subsection{Generated SprIL}
\inputminted[tabsize=4,linenos,firstnumber=0]{text}{../../src/haskell/PP-project-2016/test/simple_concurrency_gen.txt}
\subsection{Results}
\begin{minted}[tabsize=4,linenos]{text}
>>> 4
>>> 6
>>> 6
\end{minted}

\section{Simple Procedures}
\subsection{Source}
\inputminted[tabsize=4,linenos,firstnumber=1]{text}{../../src/haskell/PP-project-2016/test/simple_proc.shl}
\subsection{Generated SprIL}
\inputminted[tabsize=4,linenos,firstnumber=0]{text}{../../src/haskell/PP-project-2016/test/simple_proc_gen.txt}
\subsection{Results}
\begin{minted}[tabsize=4,linenos]{text}
>>> 2
>>> 4
>>> 0
>>> 2
\end{minted}

\section{While}
\subsection{Source}
\inputminted[tabsize=4,linenos,firstnumber=1]{text}{../../src/haskell/PP-project-2016/test/while.shl}
\subsection{Generated SprIL}
\inputminted[tabsize=4,linenos,firstnumber=0]{text}{../../src/haskell/PP-project-2016/test/while_gen.txt}
\subsection{Results}
\begin{minted}[tabsize=4,linenos]{text}
>>> 100
>>> 99
>>> 98
>>> 97
>>> 96
>>> 95
>>> 94
>>> 93
>>> 92
>>> 91
>>> 90
>>> 89
>>> 88
>>> 87
>>> 86
>>> 85
>>> 84
>>> 83
>>> 82
>>> 81
>>> 80
>>> 79
>>> 78
>>> 77
>>> 76
>>> 75
>>> 74
>>> 73
>>> 72
>>> 71
>>> 70
>>> 69
>>> 68
>>> 67
>>> 66
>>> 65
>>> 64
>>> 63
>>> 62
>>> 61
>>> 60
>>> 59
>>> 58
>>> 57
>>> 56
>>> 55
>>> 54
>>> 53
>>> 52
>>> 51
>>> 50
>>> 49
>>> 48
>>> 47
>>> 46
>>> 45
>>> 44
>>> 43
>>> 42
>>> 41
>>> 40
>>> 39
>>> 38
>>> 37
>>> 36
>>> 35
>>> 34
>>> 33
>>> 32
>>> 31
>>> 30
>>> 29
>>> 28
>>> 27
>>> 26
>>> 25
>>> 24
>>> 23
>>> 22
>>> 21
>>> 20
>>> 19
>>> 18
>>> 17
>>> 16
>>> 15
>>> 14
>>> 13
>>> 12
>>> 11
>>> 10
>>> 9
>>> 8
>>> 7
>>> 6
>>> 5
>>> 4
>>> 3
>>> 2
>>> 1
>>> 0
\end{minted}


\end{document}